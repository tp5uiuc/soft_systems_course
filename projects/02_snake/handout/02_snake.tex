% Created 2019-04-25 Thu 12:31
% Intended LaTeX compiler: pdflatex
\documentclass[11pt]{article}
\usepackage[utf8]{inputenc}
\usepackage[T1]{fontenc}
\usepackage{graphicx}
\usepackage{grffile}
\usepackage{longtable}
\usepackage{wrapfig}
\usepackage{rotating}
\usepackage[normalem]{ulem}
\usepackage{amsmath}
\usepackage{textcomp}
\usepackage{amssymb}
\usepackage{capt-of}
\usepackage{hyperref}
\usepackage{awesomebox}
\usepackage{booktabs}
\usepackage{placeins}
\usepackage{siunitx}
\usepackage{minted}
\usepackage{cleveref}
\author{Mattia Gazzola}
\date{}
\title{Project 2 + 3: Modeling and optimization of a slithering snake\\\medskip
\large ME498: Computational modeling and optimization}
\hypersetup{
 pdfauthor={Mattia Gazzola},
 pdftitle={Project 2 + 3: Modeling and optimization of a slithering snake},
 pdfkeywords={},
 pdfsubject={},
 pdfcreator={Emacs 27.0.50 (Org mode 9.2)},
 pdflang={English}}
\begin{document}

\maketitle
\textbf{Issue Date} : March 10, 2019

\textbf{Teaching Assistant} : Tejaswin Parthasarathy, \texttt{tp5@illinois.edu}

\textbf{Submission Date/Time}: Wednesday 11:59 PM, 01 May, 2019

\textbf{Submission Instructions}:
\begin{itemize}
\item You need to submit both your presentation (as \texttt{.pdf}, \texttt{.ppt[x]} or \texttt{.key}) and code
(as \texttt{.py} or \texttt{.ipynb}) on Compass, in separate archived files (see below)
\item Submit your code file(s) packaged as a \texttt{.zip} or \texttt{.tar.gz} files (please do not use
\texttt{.7z} or other formats) and name the compressed file as \texttt{<net\_id>\_code\_02}
\item Submit your presentation file(s) packaged as a \texttt{.zip} or \texttt{.tar.gz} files (please do not use
\texttt{.7z} or other formats) and name the compressed file as \texttt{<net\_id>\_slides\_02}
\end{itemize}

\newpage

\section{Guidelines}
\label{sec:orge5d2764}
General advice that pertains to solving the dynamics of an idealized
slithering snake:
\begin{enumerate}
\item Validate your code (i.e. test that it captures the physics) before doing
any optimization on your snake model. Some sensible cases for validation are
discussed later on.
\item You will need to couple the physical simulations that you develop in this
project with the optimizer---hence develop code keeping this requirement in mind.
\item Use the CMAes class/function that you developed as part of the previous
project to run the optimization campaign.
\item We will evaluate your work based on your understanding and findings, and
not on the code quality. However since you are learning \texttt{Python}, it is
always a good idea to code efficiently. Use canned algorithms in \texttt{numpy} or
\texttt{scipy} wherever possible.
\item You are free to consult any material you want (including but not limited to
resources in \cref{sec:references}). However you are expected to follow the
Student Code on academic integrity---plagiarism will not be tolerated!
\end{enumerate}

\section{A soft snake}
\label{sec:org4b530fa}
\subsection{Definition}
\label{sec:org48ac4b3}
In \texttt{Python}, implement a code for computing the mechanics of soft filaments
using Cosserat rod theory, mirroring the in-class discussion. Construct a
model of a slithering snake using this implementation and furthermore
optimize the gait of the snake for achieving maximal forward speed.

\textbf{BONUS} : Visualize any/all of your results using the \href{http://www.povray.org/}{PovRay} raytracing
library or its \texttt{Python} interface---\href{https://github.com/Zulko/vapory}{Vapory}.
\subsection{Details}
\label{sec:org83740c5}
\subsubsection{Governing equations}
\label{sec:org1ac146f}
\textbf{Cosserat rod model}
 The following are the equations that you need to implement in your code.
 Note that these equations are derived \textbf{after} appropriate simplification as
 detailed in the next subsection.
\begin{eqnarray}
\frac{\partial \mathbf{r}}{\partial t} &=& \mathbf{v}\label{eq:velfinal} \\
\frac{\partial \mathbf{d}_j}{\partial t} &=& (\mathbf{Q}^T\boldsymbol{\omega}_{\mathcal{L}}) \times \mathbf{d}_j,~~~~~j=1,2,3\label{eq:framefinal}\\
dm \cdot \frac{\partial^2 \mathbf{r}}{\partial t ^2} &=& \underbrace{\frac{\partial}{\partial \hat{s}} \left(\frac{\mathbf{Q}^T\hat{\mathbf{S}}\boldsymbol{\sigma}_{\mathcal{L}}}{1}\right)d\hat{s}}_{\text{{shear/stretch} internal force}} +\underbrace{\mathbf{F}\label{eq:linmomentfinal}}_{\text{ext. force}}\\
\frac{d\hat{\mathbf{J}}}{e} \cdot \frac{\partial \boldsymbol{\omega}_{\mathcal{L}}}{\partial t} &=& \underbrace{\frac{\partial}{\partial \hat{s}}\left(\frac{\hat{\mathbf{B}}\hat{\boldsymbol{\kappa}}_{\mathcal{L}}}{1^3}\right)d\hat{s} + \frac{\hat{\boldsymbol{\kappa}}_{\mathcal{L}}\times\hat{\mathbf{B}}\hat{\boldsymbol{\kappa}}_{\mathcal{L}}}{1^3}d\hat{s}}_{\text{{bend/twist} internal couple}}~~~+ \underbrace{\left(\mathbf{Q}\mathbf{t}\times\hat{\mathbf{S}}\boldsymbol{\sigma}_{\mathcal{L}}\right)d\hat{s}}_{\text{{shear/stretch} internal couple}}\nonumber\\
&&+ \underbrace{\mathbf{C}_{\mathcal{L}}}_{\text{ext. couple}},\label{eq:angmomentfinal}
\end{eqnarray}
where \(dm=\rho \hat{A} d\hat{s}=\rho A ds\) is the infinitesimal mass
element, and \(d\hat{\mathbf{J}}=\rho\hat{\mathbf{I}} d\hat{s}\) is the
infinitesimal mass second moment of inertia. Note that both
initial and boundary conditions are problem dependent.

\noindent\rule{1\textwidth}{0.01pt}

\noindent \textbf{Numerical discretization}
Numerically discretize the above equations according to the discussion in
class, using nodal masses and elemental frames. Implement frame rotations/
transformations using the Rodrigues rotation formula, using code from the
hands-on session on rotations. Utilize the energy-preserving second-order
(in position and velocity) position-Verlet scheme, following the details on
the hands-on session on timestepping schemes.

\subsubsection{Simplifications to the governing equations}
\label{sec:orgb411169}
\begin{itemize}
\item The first simplification performed above involves a crude
approximation---we simply remove the Lagrangian transport and unsteady
dilatation terms from the angular momentum evolution equation\cref{eq:angmomentfinal}.
\item The second simplification comes from explicitly constraining \(e = 1\).
This means you \textbf{DO NOT} need quantities in the reference and current
configuration---they are always identical. This means that you do not
need hatted (\(\hat{.}\)) and non-hatted matrices/quantities.
Implementation-wise, you will define constant matrices and reuse them
throughout all the elements. We note that this approximation is only
valid in the specific example of slithering snake that we are interested in.
\item As we will later see, rolling/lateral friction is neglected.
\end{itemize}
\subsubsection{Validation}
\label{sec:org120c7fa}
Once you are confident that all your operators (\texttt{Python}  functions) are
performing the right task, it is prudent to validate your code against
physical cases with known analytical (or numerical/experimental) solutions.
One such example is the deformation of a cantilever beam under an applied
load (and you are free to choose other cases too, if you so wish).

In this problem, you clamp one end of the beam \(\hat{s}=0\) , while applying a load \(F\) to the free end of the beam \(\hat{s}=\hat{L}\). This is shown in \cref{p1_fig} below, taken
 from the first link in \cref{sec:references}. In this case, analytical solutions dervied from the Timoshenko (and
 Euler-Bernoulli) theory exists, which relies on the assumption of small
 deflections, so that the horizontal coordinate \(x\) along the direction
 \(\mathbf{k}\) can be approximated by the arc-length \(\hat{s}\).

\begin{figure}[htbp]
\centering
\includegraphics[width=0.9\textwidth]{images/figure_11.pdf}
\caption{\label{p1_fig}
Validation---Deformation of a cantilever beam}
\end{figure}
Considering the rod has length
 \(\hat{L}\), constant cross sectional area \(\hat{A}\) , area second moment of
 inertia about the axis \(\mathbf{j}=\mathbf{k}\times\mathbf{i}\) to be
 \(\hat{I}_1\), Young's and shear moduli \(E\) and \(G\), the analytical solution is

\[ y=-\frac{3F}{4\hat{A}G}\hat{s} -
	\frac{F\hat{L}}{2E\hat{I}_1}\hat{s}^2 + \frac{F}{6E\hat{I}_1}\hat{s}^3 \]

If the shear modulus \(G\) approaches infinity or if the ratio
\(3E\hat{I}_1/(4\hat{L}^2\hat{A}G)\ll 1\), then the Timoshenko model in the
static case reduces to the Euler-Bernoulli approximation, yielding
\[ y= - \frac{F\hat{L}}{2E\hat{I}_1}\hat{s}^2 +
	\frac{F}{6E\hat{I}_1}\hat{s}^3 \]

You can then compare your numerical model with these results by carrying out
simulations of the cantilever beam shown in \cref{p1_fig}, with generous number
of elements and an appropriate \(dt\). You should recover the results
obtained from Timoshenko theory, shown in \cref{p1_fig} for these parameters:

\begin{table}[htbp]
\caption{\label{timoshenko_params}
Parameters for the Timoshenko beam validation}
\centering
\begin{tabular}{ll}
\toprule
Parameters & Value\\
\midrule
Rod density \(\rho\) & \(\SI{5e3}{\kg \m^{-3}}\)\\
Young's modulus \(E\) & \(\SI{1e6}{\Pa}\)\\
Shear modulus \(G\) & \(\SI{1e4}{\Pa}\)\\
Downward force \(F\) & \(\SI{15}{\N}\)\\
Rod Length \(L\) & \(\SI{3}{\m}\)\\
Rod radius \(r\) & \(\SI{0.25}{\m}\)\\
Dissipation constant \(\gamma\) & \(\SI{0.1}{\kg\per\m\per\second}\)\\
Simulation time \(T\) & \(\SI{5e3}{\second}\)\\
Number of discretization points \(n\) & 100\\
Time step \(dt\) & \(\SI{3e-4}{\second}\)\\
\bottomrule
\end{tabular}
\end{table}

Notice that if you change one of these parameters such that \(3E\hat{I}_1/(4\hat{L}^2\hat{A}G)\ll 1\), you should also recover the
results of the Euler-Bernoulli theory (Say by setting \(E = \textcolor{red}{\frac{3}{2}G} = \SI{1e5}{\Pa}\) ).

\noindent\rule{1\textwidth}{0.01pt}
\textbf{Initial conditions}: Notice that in this case, the initial
condition constrains the rod to be straight, with its axis (and hence all
elemental frames) pointing in the \(\mathbf{i}\) direction. Setting the spatial location
of the equispaced nodes/frames initially then, is pretty straightforward. Additionally at the
start, all nodes have translational and angular velocities set to \(\mathbf{0}\), in the
appropriate units.

\noindent\rule{1\textwidth}{0.01pt}
\textbf{Boundary conditions}: Notice that in this case, the boundary
condition constrains the elemental frame (and its angular velocity) at \(\hat{s} = 0.0\) to retain its initial configuration. Furthermore at this
location, the node is time invariant---hence its location is fixed, and its
velocity always \(\mathbf{0}\).
\subsubsection{Towards a slithering snake}
\label{sec:orgedb9da9}
\textbf{Muscular activity} To model muscular activity, we express it as torques
 acting along the body. The magnitude \(A_m\) of this torque is a traveling
 wave propagating head to tail along the filament

\[ A_m=\beta_m(\hat{s})\cdot\sin\left(\frac{2\pi}{T_m} t +
											\frac{2\pi}{\lambda_m}
											\hat{s}\right) \]
where \(t\) is time, \(T_m\) and \(\lambda_m\) are,
respectively, the activation period and wavelength. In the equation above,
the amplitude of the traveling wave is represented by the cubic B-spline \(\beta(\hat{s})\) characterized by \(N_m\) control points \((\hat{S}_i,\beta_i)\) with \(i=0,\dots,N_m-1\), as illustrated in
\cref{fig_spline}. The first and last control points are fixed so that \((\hat{s}_0,\beta_0)=(0,0)\)
and \((\hat{s}_{N_m-1},\beta_{N_m-1})=(\hat{L},0)\), therefore assuming the
ends of the deforming body to be free.

\begin{figure}[htbp]
\centering
\includegraphics[width=0.9\textwidth]{images/figure_03.pdf}
\caption{\label{fig_spline}
B-spline parametrization for modeling muscular activity using torques. We exhibit the case with \(N_m = 8\) here.}
\end{figure}

We then prescribe this muscular activity as an internal torque activation of
the form
\[ \boldsymbol{\tau}^m_{\mathcal{L}} = \mathbf{Q}(A_m\mathbf{d}_1) \]
assuming \(\mathbf{d}_2\) and \(\mathbf{d}_3\) to be coplanar to the ground.
This contribution is directly added to the internal torque
\(\boldsymbol{\tau}_{\mathcal{L}}\) resulting from solving the Cosserat equations.

The cubic B-spline function with the appropriate boundary conditions has
been implemented for you and is available as a function from the scriptfile
\texttt{b\_spline.py}. A typical use-case is shown in the code \cref{list_spline}
below, which produces the spline shown in \cref{python_spline}.


\begin{listing}[htbp]
\begin{minted}[,fontsize=\scriptsize]{python}
import numpy as np
from matplotlib import pyplot as plt
from bspline import snake_bspline

# Length of centerline
l_centre = 1.0
# Non-zero coefficients of spline, set by you
t_coeff = np.array([0.1, 0.3, 0.15, 0.22, 0.25, 0.1])
# See function documentation for more details
my_spline, ctr_pts, ctr_coeffs = snake_bspline(t_coeff, keep_pts=True)

s = np.linspace(0.0, l_centre, 200)

# Figure beautification
fig, ax = plt.subplots(figsize=(8,2))
ax.set_aspect('equal')
ax.set_xlim(0.0, 1.0)
ax.set_ylim(0.0, 0.4)
ax.set_xlabel(r'$\hat{s}$')
ax.set_ylabel(r'$\beta_m(\hat{s})$')

# Plot the spline along as function of centerline
ax.plot(s, my_spline(s))

# Plot the control points of the spline too
ax.plot(ctr_pts, ctr_coeffs, 'kx')

# Export
FILE_NAME = 'images/snake_spline.pdf'
fig.savefig(FILE_NAME, bbox_inches='tight')
FILE_NAME
\end{minted}
\caption{\label{list_spline}
B-spline code snippet}
\end{listing}

\begin{figure}[htbp]
\centering
\includegraphics[width=1.0\textwidth]{images/snake_spline.pdf}
\caption{\label{python_spline}
The spline generated by the script \texttt{b\_spline.py}}
\end{figure}

\noindent\rule{1\textwidth}{0.01pt}
\textbf{Contact with the wall}: The wall (or ground) contact is modeled as an external
response force experienced by the rod \(\mathbf{F}^w_{\perp}\) that balances
the sum of all forces \(\mathbf{F}_{\perp}\) that push the rod against the
wall, and is complemented by other two components which help prevent
possible interpenetration due to numerical errors. The interpenetration
distance \(\epsilon\) triggers a normal elastic response proportional to the
stiffness of the wall \(k_{w}\), while a dissipative term related to the
normal velocity component of the filament with respect to the wall accounts
for a damping force proportional to \(\gamma_w\), so that the overall wall
response is
\[ \mathbf{F}^w_{\perp}= H(\epsilon)\cdot(-\mathbf{F}_{\perp} +
	k_w\epsilon-\gamma_w\mathbf{v}\cdot
	\mathbf{u}^w_{\perp})\mathbf{u}^w_{\perp} \]
where \(H(\epsilon)\) denotes the Heaviside function and ensures that a wall
force is produced only in case of contact (\(\epsilon\ge0\)). Here
\(\mathbf{u}^w_{\perp}\) is the boundary outward normal (evaluated at the
contact point, that is the contact location for which the normal passes
through the center of mass of the element), and \(k_w\) and \(\gamma_w\) are,
respectively, the wall stiffness and dissipation coefficients.

Once wall contact is modeled, you can run some test cases to see
whether it works. As the response is linear, when \(\epsilon > 0\),
consider running the following three cases while recording the force on the
cylinder:
\begin{itemize}
\item A rod with nodal mass \(dm\) resting horizontally on the ground (which
is also at rest), when
uniform gravity \(g = \SI{9.81}{\m\per\s^2}\) acts in the vertical
direction (i.e, in the wall normal direction). In
this case, the wall force should equal the force due to gravity for static
equilibrium, i.e. \(\mathbf{F}^w_{\perp}=\)
\item Now turn gravity off in the scenario above, but initialize the rod such
that it has some interpenetration \(\epsilon\) with the ground (once
again, in the wall normal direction). If the
wall stiffness is \(k_w\), then the instantaneous wall response should
record in your solver as \(k_w \epsilon\).
\item To check the damping force, we envision two cases shown below. In both
cases gravity is turned off:
\begin{itemize}
\item The rod lies on the ground similar to the first case, but it now has a
uniform velocity \(v\) in the ground coplanar direction (say
horizontal). In this case, the wall response should record zero (or
values close to zero).
\item If however, the uniform velocity \(v\) now acts in the wall normal
direction and tries to penetrate the rod into the ground, then the
instantaneous wall normal response should read \(\gamma_w v\) (or
values close to the same, accounting for interpenetration \(\epsilon\) ).
\end{itemize}
\end{itemize}

\noindent\rule{1\textwidth}{0.01pt}
\textbf{Anisotropic friction}: The modeling of friction should closely follow
the in-class discussions. Once the isotropic friction law is setup using
the Amonton--Coloumb law, anisotropy can be included by changing the
friction coefficients based on the direction of locomotion.
For this project, you can \textbf{neglect} lateral/rolling friction.

\subsubsection{The slithering snake}
\label{sec:org52aed0e}
With all the components in place, we can assemble them together to model a
snake. For this case, the muscular activity
is modeled as an internal torque, calculated as a parametrized B-spline, as mentioned
before. We first discuss initial and boundary conditions:

\noindent\rule{1\textwidth}{0.01pt}
\textbf{Initial conditions} The rod representing the snake is initialized coplanar
to the ground, with equispaced nodes along the forward direction. At the start, \(\mathbf{d}_1\) is assumed
to point in the wall-normal direction and so \(\mathbf{d}_2, \mathbf{d}_3\) point in the coplanar direction. We also remind you that \(\mathbf{d}_3\) is
set to points along the body centerline coordinate, at the start. All nodal
translational velocities and elemental angular velocities are initialized as zero.

\noindent\rule{1\textwidth}{0.01pt}
\textbf{Boundary conditions} With the torque profile imposed by the B-spline, we
need not specify boundary conditions on the snake (aka Free boundary conditions).

\noindent\rule{1\textwidth}{0.01pt}
\textbf{Muscle activity} We consider a \textbf{six} parameter B-spline,
with \(\beta_{i=0,5}=0\) to model the muscle activity.

\noindent\rule{1\textwidth}{0.01pt}
\textbf{Additional validation} If you are not confident with your snake model, you
can refer to the first link in \cref{sec:references} for more validation cases or alternatively
ask the TA.

\subsubsection{Gait optimization for maximal forward velocity}
\label{sec:orgf2ace08}
We are now (almost) ready to tackle the optimization problem of finding the
maximal forward velocity for a model snake. The code setup, initial and
boundary conditions follow from the previous section, including the
six coefficient spline parameterization.

The rod parameters for this case are given in \cref{tab_opt_snake_params}.

\begin{table}[htbp]
\caption{\label{tab_opt_snake_params}
Parameters for the snake to be optimized for maximal forward velocity}
\centering
\begin{tabular}{ll}
\toprule
Parameters & Value\\
\midrule
Rod density \(\rho\) & \(\textcolor{red}{\SI{1e3}{\kg \m^{-3}}}\)\\
Young's modulus \(E\) & \(\SI{1e7}{\Pa}\)\\
Shear modulus \(G\) & \(2E/3\;\si{\Pa}\)\\
Shear/Stretch matrix \(\mathbf{S}\) & diag\((4GA/3, 4GA/3, \textcolor{red}{EA}) \si{\N\per\m^2}\)\\
Bend/Twist matrix \(\mathbf{B}\) & diag\((EI_1, EI_2, GI_3) \si{\N\per\m^2}\)\\
Rod length \(L\) & \(\SI{1}{\m}\)\\
Rod radius \(r\) & \(\SI{0.025}{\m}\)\\
Muscular activation period \(T_m\) & \(\SI{1}{\second}\)\\
Dissipation constant \(\gamma\) & \(\SI{5}{\kg\per\m\per\second}\)\\
Acceleration due to gravity normal to ground \(g\) & \SI{9.81}{\m \per \s^2}\\
Forward kinetic friction coefficient\(\mu^f_k\) & \(1.019368\)\\
Backward kinetic friction coefficient \(\mu^b_k\) & \(1.5 \cdot  \mu^f_k\)\\
Forward static friction coefficient\(\mu^f_s\) & \(2 \cdot  \mu^f_k\)\\
Backward static friction coefficient \(\mu^b_s\) & \(1.5 \cdot  \mu^f_s\)\\
Friction threshold velocity \(v_{\epsilon}\) & \SI{1e-8}{\m\per\s}\\
Ground stiffness \(k_w\) & \SI{1}{\kg \per \s^2}\\
Ground viscous dissipation & \SI{1e-6}{\kg \per \s}\\
Number of discretization points \(n\) & 50\\
Time step \(dt\) & \(\SI{2.5e-5}{\second}\)\\
\bottomrule
\end{tabular}
\end{table}

\noindent\rule{1\textwidth}{0.01pt}
\textbf{Coupling with CMAes} : With these parameters, you can now run an optimization
 campaign using CMAes, to find an optimal gait that maximizes the forward
 velocity \(v^{\text{fwd}}_{\text{max}}\) over one activation cycle. That
 is, you are required to find the spline coefficients and wavelength:
 \[ \beta_{i} \quad i=1,2,3,4 \quad \text{and} \quad \lambda_m \]
 with \(\beta_{i=0} = \beta_{i=5} = \SI{0}{\N\m}\) identically. Think
 about the fitness function for this problem, and any bounds that you would
 like to place on the parameters to be optimized (for example, we ran it
 with \(|\beta|^{\text{max}}_{i=0,\dots,5} = \SI{50}{\N\m}\) ). The
 optimized parameters in our case (including lateral friction) were
 \[ \beta_{i=0,\dots,5}=\{0,17.4, 48.5, 5.4, 14.7, 0\} \quad \text{and}
     \quad \lambda_m = \SI{0.97}{m} \]
 which gives an average forward velocity of \(v^{\text{fwd}}_{\text{max}}\simeq \SI{0.6}{\m\per\s}\), which
 compares well to real-life snakes*. This
 is shown in \cref{fig_opt_snake}, and should (ideally) not be far from the final velocity
 that your implementation gives as well (considering that you are making a
 lot of assumptions). Once again, to encourage comparison
 with our results, the forward and lateral velocities of the optimal snake
 is attached in the file \texttt{optimized\_snake.dat}, and can be read into your
 \texttt{Python} environment, using \texttt{numpy}'s \texttt{loadtxt} function, as shown in \cref{list_snakedata}.

\begin{listing}[htbp]
\begin{minted}[,fontsize=\scriptsize]{python}
import numpy as np

# my_data is a (x, 3) time series data
# The first columnd my_data[:,0] contains the time
# The second columnd my_data[:,1] contains the forward snake velocity
# The third columnd my_data[:,2] contains the lateral snake velocity
my_data = np.loadtxt('optimized_snake.dat')
\end{minted}
\caption{\label{list_snakedata}
Importing the snake dataset}
\end{listing}

\begin{figure}[htbp]
\centering
\includegraphics[width=1.0\textwidth]{images/figure_13.pdf}
\caption{\label{fig_opt_snake}
Optimal lateral undulation gait. (a, b, c, d) Instances at different times of a snake characterized by the identified optimal gait. (e) Evolution of the fitness function \(f=v^{\text{fwd}}_{\text{max}}\) as function of the number of generations produced by CMA-ES. Solid blue, solid red and dashed black lines represent, respectively, the evolution of \(f\) corresponding to the best solution, the best solution within the current generation, and the mean generation value. (f) Scaled forward (red) and lateral (blue) center of mass velocities versus normalized time. (g) Gait envelope over one oscillation period \(T_m\). Red lines represent head and tail displacement in time.}
\end{figure}

\subsection{Expected submission}
\label{sec:org12eef78}
We expect you to submit your code and presentation. In your presentation,
please try and include all your validation cases, information on your
optimization campaign (CMAes fitness function, bounds on parameters, time
interval for optimization etc), performance of CMA on this problem and finally
the results obtained. Other information such as timing data from your code
(i.e. time for a single function evaluation) etc. can also be included.
\section{A soft X}
\label{sec:orgf6d5f16}
{\large \textbf{This section is compulsary for students taking the course for four credit
hours and is optional for those taking the course for three credit hours}}.
\subsection{Solve for X}
\label{sec:org0e81f38}
Think of an physical problem/application in which the modeling capabilities
that you learnt in this course can come in handy, and design, using Cosserat
rod(s), a setup that can be used to study/solve the problem you have in mind.
This problem/application can be inspired from your research (or) even
something you are really curious about (a few examples will be given in the
class, but we will brainstorm with you about potential project ideas and
their feasibility).
\subsection{Optimize for X}
\label{sec:org554c477}
Once you have a preliminary design that partially/fully solves your problem,
you need to see whether you can do \emph{better}. You are then expected to setup an
inverse design problem, define what's \emph{better} in your case and use the
optimization techniques that you have learnt thus far, to \emph{evolve} new
designs.
\subsection{Understand X}
\label{sec:orgf41acd1}
After you have arrived at a \emph{good} design, try and understand what makes it
\emph{good}. While this may not be straightforward in all cases, tracking the designs
that CMAes evolves can give you some intution as to why your design may be optimal.

\section{The following resouces may prove useful:}
\label{sec:references}
\begin{itemize}
\item Paper describing the governing equations, numerical algorithm and optimization
of a slithering snake, found \href{https://royalsocietypublishing.org/doi/full/10.1098/rsos.171628}{in this link}.
\item The CMA-ES tutorial @ Arxiv, found \href{https://arxiv.org/pdf/1604.00772.pdf}{here}
\item More information on timestepping schemes found \href{https://cg.informatik.uni-freiburg.de/course\_notes/sim\_02\_particles.pdf}{at this link}
\item \href{http://young.physics.ucsc.edu/115/leapfrog.pdf}{This link} on a short but gentle introduction to symplectic time integration
schemes accompanied by \href{http://www2.math.ethz.ch/education/bachelor/seminars/fs2008/nas/crivelli.pdf}{this link} that compares many other schemes to the same.
\end{itemize}
\section{Compass Instructions}
\label{sec:org4884744}
\textbf{Project 2}
In this project, you will implement a computational soft mechanics code (in Python) and construct a near-realistic model of a slithering snake. You will also hook it up to a stochastic optimization algorithm (CMAes) to find a gait that maximizes the forward speed, under some given conditions.

\noindent \textbf{Project 3}
Students taking the course for four credit hours additionally need to think of a model problem/application which they can solve using the implemented code. You also need to develop an inverse design study to find a better (optimal?) solution that tackles the same problem, and try and determine why the solution is optimal.

More information enclosed in the attached files.
\end{document}
