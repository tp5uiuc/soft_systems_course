% Created 2019-05-19 Sun 00:42
% Intended LaTeX compiler: pdflatex
\documentclass[presentation]{beamer}
\usepackage[utf8]{inputenc}
\usepackage[T1]{fontenc}
\usepackage{graphicx}
\usepackage{grffile}
\usepackage{longtable}
\usepackage{wrapfig}
\usepackage{rotating}
\usepackage[normalem]{ulem}
\usepackage{amsmath}
\usepackage{textcomp}
\usepackage{amssymb}
\usepackage{capt-of}
\usepackage{hyperref}
\usepackage{awesomebox}
\usepackage{booktabs}
\usepackage{placeins}
\usepackage{siunitx}
\usepackage{minted}
\usetheme[progressbar=frametitle]{metropolis}
\usepackage{tikz}
\usepackage{tikz-3dplot}
\usepackage{spot}
\usepackage{pgfplots}
\usetikzlibrary{arrows.meta}
\pgfplotsset{compat=1.16}
\newcommand{\gv}[1]{\ensuremath{\mbox{\boldmath$ #1 $}}}
\newcommand{\bv}[1]{\ensuremath{\mathbf{#1}}}
\newcommand{\norm}[1]{\left\lVert#1\right\rVert}
\newcommand{\abs}[1]{\left\lvert#1\right\rvert}
\newcommand{\bigqm}[1][1]{\text{\larger[#1]{\text{?}}}}
\newcommand{\order}[1]{\mathcal O \left( #1 \right)} % order of magnitude
\definecolor{scarlet}{rgb}{1.0, 0.13, 0.0}
\definecolor{shamrockgreen}{rgb}{0.0, 0.62, 0.38}
\definecolor{royalblue}{rgb}{0.25, 0.41, 0.88}
\definecolor{metropolisorange}{RGB}{235,129,27}
\definecolor{metropolisblue}{RGB}{35,55,59}
\usetheme{default}
\author{\emph{Tejaswin Parthasarathy}, Mattia Gazzola}
\date{\today}
\title{Elastica : Coordinate/Frame transformations}
\subtitle{ME498: Comp. modeling \& optimization}
\hypersetup{
 pdfauthor={\emph{Tejaswin Parthasarathy}, Mattia Gazzola},
 pdftitle={Elastica : Coordinate/Frame transformations},
 pdfkeywords={},
 pdfsubject={},
 pdfcreator={Emacs 27.0.50 (Org mode 9.2)},
 pdflang={English}}
\begin{document}

\maketitle
\section{Coordinate/Frame transformations}
\label{sec:orgd94388e}
\begin{frame}[label={sec:org59c56d1}]{Motivation}
\[ \gv{x}_{\mathcal{L}} = \bv{Q}\gv{x} \]
\[ \scalebox{5}{\textbf{?}} \]
\[ \frac{\partial \bv{d}_j}{\partial t} = \left( \bv{Q}^T
   \omega_{\mathcal{L}}\right) \times \bv{d}_j \]
\[ \scalebox{5}{\textbf{?}} \]
\end{frame}

\begin{frame}[label={sec:orge85effd}]{Motivation}
\begin{columns}
\begin{column}{0.7\columnwidth}
\begin{itemize}
\item To convert between arbitrary spaces, e.g. world space and other spaces (in
graphics) or Eulerian frame to Lagrangian frame in physics
\item To convert between coordinates that are more ``natural'' to the system under
observation---e.g. complex numbers can be naturally represented in polar,
rather than cartesian coordinates
\end{itemize}
\begin{itemize}
\item Transformations can be applied to points, vectors etc.
\end{itemize}
\end{column}
\begin{column}{0.5\columnwidth}
\begin{center}
	\begin{tikzpicture}
	\begin{axis}[
		width=1\textwidth,
		height=0.8\textheight,
		xmin=-1.5,
		xmax=4.5,
		ymin=-4.5,
		ymax=4.5,
		axis equal,
		axis lines=middle,
		grid=major,
		xlabel=$\Re(z)$,
		ylabel=$\Im(z)$,
		disabledatascaling]
		% https://tex.stackexchange.com/questions/27279/how-to-make-an-arrow-bigger-and-change-its-color-in-tikz/27287#27287
		\addplot [arrows={-latex[scale=4]}, thick, color=metropolisorange] coordinates { (0,0) (2,3) } node [right] {$2 + 3i$};
		\addplot [arrows={-latex[scale=4]}, thick, color=metropolisblue] coordinates { (0,0) (3,-2) } node [below] {$3 - 2i$};
	\end{axis}
	\end{tikzpicture}
\end{center}
\end{column}
\end{columns}
\end{frame}
\end{document}
