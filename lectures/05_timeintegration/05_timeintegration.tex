% Created 2019-03-14 Thu 11:54
% Intended LaTeX compiler: pdflatex
\documentclass[presentation]{beamer}
\usepackage[utf8]{inputenc}
\usepackage[T1]{fontenc}
\usepackage{graphicx}
\usepackage{grffile}
\usepackage{longtable}
\usepackage{wrapfig}
\usepackage{rotating}
\usepackage[normalem]{ulem}
\usepackage{amsmath}
\usepackage{textcomp}
\usepackage{amssymb}
\usepackage{capt-of}
\usepackage{hyperref}
\usepackage{awesomebox}
\usepackage{booktabs}
\usepackage{placeins}
\usepackage{siunitx}
\usepackage{minted}
\usetheme[progressbar=frametitle]{metropolis}
\usepackage{tikz}
\usepackage{tikz-3dplot}
\usepackage{spot}
\newcommand{\gv}[1]{\ensuremath{\mbox{\boldmath$ #1 $}}}
\newcommand{\bv}[1]{\ensuremath{\mathbf{#1}}}
\newcommand{\norm}[1]{\left\lVert#1\right\rVert}
\newcommand{\abs}[1]{\left\lvert#1\right\rvert}
\newcommand{\bigqm}[1][1]{\text{\larger[#1]{\text{?}}}}
\newcommand{\order}[1]{\mathcal O \left( #1 \right)} % order of magnitude
\definecolor{scarlet}{rgb}{1.0, 0.13, 0.0}
\definecolor{shamrockgreen}{rgb}{0.0, 0.62, 0.38}
\definecolor{royalblue}{rgb}{0.25, 0.41, 0.88}
\usetheme{default}
\author{\emph{Tejaswin Parthasarathy}, Mattia Gazzola}
\date{\today}
\title{Elastica : Timesteppers}
\subtitle{ME498: Comp. modeling \& optimization}
\hypersetup{
 pdfauthor={\emph{Tejaswin Parthasarathy}, Mattia Gazzola},
 pdftitle={Elastica : Timesteppers},
 pdfkeywords={},
 pdfsubject={},
 pdfcreator={Emacs 27.0.50 (Org mode 9.2)},
 pdflang={English}}
\begin{document}

\maketitle
\section{Time-marching algorithms}
\label{sec:org3b086a4}
\begin{frame}[label={sec:org5728d5d}]{Motivation}
\begin{example}[Solve the following ODE]
\[ \frac{dx}{dt} = 2x \quad x(0) = 1 \]
\end{example}
\begin{block}{We can!}
\[ x(t) = e^{2t}\]
\end{block}
\end{frame}

\begin{frame}[label={sec:org464cc33}]{Motivation}
\begin{example}[Solve the following ODE]
\[ \frac{dx}{dt} = \sin(\cos x^{\frac{4}{3}}) + 4\sin^2(t) \quad x(0) = 1 \]
\end{example}

\begin{block}<2->{We can!}
Use a time-marching algorithm that can solve the above equation, albeit numerically
\end{block}
\end{frame}
\begin{frame}[label={sec:orgb5ff825}]{Introduction}
\begin{itemize}
\item As seen in the last lecture, all our (temporal/spatial) rate of frame change
vectors (that effect rotations) are precisely of this form
\end{itemize}

More generally, we can solve problems of the form
\[ \frac{\partial \mathbf{u}}{\partial t} = \mathbf{F}(\mathbf{u}, t) \]
which is a partial differential equation (PDE), wherein \(\mathbf{F}\) is
any arbitrary function.
\end{frame}
\begin{frame}[label={sec:orgaae2e48}]{Introduction}
\begin{itemize}
\item We investigate three different (classes of) time-marching algorithms for
autonomous problems (?!):
\begin{itemize}
\item Euler method (or Euler forward)
\item Runge-Kutta-4/RK4 (multi-stage methods)
\item Position Verlet (symplectic, area preserving) integrators
\end{itemize}
\item We develop time marching methods that compute approximations to \(u(t)\)
at specfic time points, \(t_0, t_1, \cdots, t_n\).
\begin{itemize}
\item We only consider a uniform timestep size \(dt  \rightarrow t^n = n \cdot
       dt\).
\end{itemize}
\item Finally, we \emph{compare} these methods based on general and problem-specific properties\ldots{}
\end{itemize}
\end{frame}
\begin{frame}[label={sec:orgc55516c}]{Deriving Euler's forward method\footnote{Wikimedia}}
\begin{columns}
\begin{column}{0.5\columnwidth}
\begin{itemize}
\item Simplest timestepping scheme
\item First-order approximation at time \(t_0\)
\begin{itemize}
\item Geometrical description
\item Taylor series expansion
\end{itemize}
\end{itemize}
\end{column}
\begin{column}{0.4\columnwidth}
\begin{figure}[htbp]
\centering
\includegraphics[width=0.8\textwidth]{images/euler_geometry.pdf}
\caption{Geometrical desciption of Euler's method}
\end{figure}
\end{column}
\end{columns}

\[ u(t_{0}+dt)=u(t_{0})+dtu'(t_{0})+{\frac {1}{2}}dt^{2}u''(t_{0})+O(dt^{3}). \]
\begin{itemize}
\item First order because local slope approximation is \(\order{dt}\)
\end{itemize}
\end{frame}

\begin{frame}[label={sec:org88c8539}]{General time stepping schemes}
\begin{itemize}
\item General schemes approximate the next iterate \(t^{n+1}\) using:
\end{itemize}
\[ \mathbf{u}^{n+1} = \sum_{i=0}^{k} \alpha_i \mathbf{u}^{n-i} + \sum_{j=0}^{r} \beta_j \frac{\partial \mathbf{u}^{n-j}}{\partial t} \]
which for \(k=1\) and \(r=0\) looks something along these lines:
\[ \mathbf{u}^{n+1} = \alpha_0 \mathbf{u}^{n} + \alpha_1 \mathbf{u}^{n-1} + \beta_0 \frac{\partial \mathbf{u}^{n}}{\partial t} \]
\begin{itemize}
\item Derivation of schemes other than Euler method follow a similar line of reasoning, while
details vary\footnote{By a \alert{lot}}
\end{itemize}
\end{frame}
\begin{frame}[label={sec:org4d43e43}]{Some time stepping schemes\footnote{Oleg Alexandrov, Public Domain, Wikimedia}}
With \(\dot{x} = f(x)\),
\begin{block}{Euler forward}
\[ x^{n+1} = x^{n} + f(x^{n})dt \]
\end{block}
\begin{columns}
\begin{column}{0.5\columnwidth}
\begin{block}{Midpoint method}
\begin{equation*}
\begin{aligned}
x^{*}&= x^{n} + f({x}^{n}) \cdot \frac{dt}{2} \\
x^{n+1} &= x + f({x}^{*}) \cdot dt \\
\end{aligned}
\end{equation*}
\end{block}
\end{column}
\begin{column}{0.4\columnwidth}
\begin{figure}[htbp]
\centering
\includegraphics[width=0.8\textwidth]{images/midpoint_method.png}
\caption{Geometrical desciption of the midpoint method}
\end{figure}
\end{column}
\end{columns}
\end{frame}

\begin{frame}[label={sec:orge536a50}]{Some time stepping schemes}
\begin{block}{Runge Kutta-4}
\begin{equation*}
\begin{aligned}
{k}_1 &= {f}({x}^{n}) \cdot dt \\
{k}_2 &= {f}({x}^{n} + 0.5 \cdot {k}_1)\cdot dt \\
{k}_3 &= {f}({x}^{n} + 0.5 \cdot {k}_2)\cdot dt \\
{k}_4 &= {f}({x}^{n} + 0.5 \cdot {k}_3)\cdot dt \\
{x}^{n+1} &= {x}^{n} + \frac{{k}_1+2{k}_2+2{k}_3+{k}_4}{6}
\end{aligned}
\end{equation*}
\end{block}
\begin{block}{Position/Velocity Verlet}
\begin{itemize}
\item Later on we introduce these two schemes in the context of integrating
second order ODEs
\end{itemize}
\end{block}
\end{frame}
\begin{frame}[label={sec:org3b00ad4}]{Function evaluations}
\begin{itemize}
\item Our first attempt at comparing schemes is the number of functional
evaluations for one time step\ldots{}
\item Why? \(f(x)\) can be expensive to evaluate (e.g. calculating the effect
of the energy diffusion on millions to billions of
grid points in an astrophysical simulation)
\item Comparing schemes,
\end{itemize}
\begin{table}[htbp]
\caption{\label{tab_sym_snake_params}
Number of function evaluations for schemes}
\centering
\begin{tabular}{lr}
\toprule
Scheme & \(n[f(x)]\)\\
\midrule
Euler & 1\\
Midpoint & 2\\
RK4 & 4\\
Verlet* & 1\\
\bottomrule
\end{tabular}
\end{table}
\end{frame}
\begin{frame}[label={sec:org1ebdd23}]{Convergence}
\begin{definition}[Order of accuracy]
The numerical solution \(\mathbf{u}\) is said to be \(n^{\text{th}}\)-order
accurate if the error, \(e(dt):=\lVert\tilde{\mathbf{u}}-\mathbf{u} \rVert\)
is proportional to the step-size \(dt\), to the \(n^{\text{th}}\) power. That
is
\[ e(dt)=\lVert\tilde{\mathbf{u}}-\mathbf{u} \rVert\leq C(dt)^{n} \]
where the constant \(C\) is independent of \(dt\) and usually depends on
the solution \(\mathbf{u}\)
\end{definition}
 In the big O notation an \(n^{\text{th}}\)-order accurate numerical method
 is notated as
\[ \lVert\tilde{\mathbf{u}}-\mathbf{u} \rVert = \order{h^n}\]
\end{frame}
\begin{frame}[label={sec:org8dd0942}]{Convergence : Importance}
\begin{columns}
\begin{column}{0.4\columnwidth}
\begin{block}{First order}
\begin{center}
\begin{tabular}{ll}
\toprule
\(dt\) & \(e(dt)\)\\
\midrule
10\textsuperscript{-1} & 1\\
10\textsuperscript{-2} & 10\textsuperscript{-1}\\
10\textsuperscript{-3} & 10\textsuperscript{-2}\\
10\textsuperscript{-4} & 10\textsuperscript{-3}\\
10\textsuperscript{-5} & 10\textsuperscript{-4}\\
\bottomrule
\end{tabular}
\end{center}
\end{block}
\end{column}
\begin{column}{0.4\columnwidth}
\begin{block}{Second order}
\begin{center}
\begin{tabular}{ll}
\toprule
\(dt\) & \(e(dt)\)\\
\midrule
10\textsuperscript{-1} & 1\\
10\textsuperscript{-2} & 10\textsuperscript{-2}\\
10\textsuperscript{-3} & 10\textsuperscript{-4}\\
10\textsuperscript{-4} & 10\textsuperscript{-6}\\
10\textsuperscript{-5} & 10\textsuperscript{-8}\\
\bottomrule
\end{tabular}
\end{center}
\end{block}
\end{column}
\begin{column}{0.4\columnwidth}
\begin{block}{Fourth order}
\begin{center}
\begin{tabular}{ll}
\toprule
\(dt\) & \(e(dt)\)\\
\midrule
10\textsuperscript{-1} & 1\\
10\textsuperscript{-2} & 10\textsuperscript{-4}\\
10\textsuperscript{-3} & 10\textsuperscript{-8}\\
10\textsuperscript{-4} & 10\textsuperscript{-12}\\
10\textsuperscript{-5} & 10\textsuperscript{-16}\\
\bottomrule
\end{tabular}
\end{center}
\end{block}
\end{column}
\end{columns}
\end{frame}
\begin{frame}[label={sec:orgfeb0071}]{Convergence : Implementation}
\begin{block}{Model problem definition}
Let's solve this problem, and test out methods for convergence:
\[ \frac{dy}{dt} = -y \quad,\quad  y(0) = 1 \]
which as we know has the analytical solution \(\tilde{y}(t) = e^{-t}\) \(\rightarrow\)
error known at every \(dt\)

Notice:
\begin{itemize}
\item We choose a simple problem to understand performance/convergence
\begin{itemize}
\item More complicated problems usually follow suit
\end{itemize}
\item We are solving an eigenvalue problem, just like the last lecture (rotations)
\end{itemize}

\alert{ACTIVITY}
\end{block}
\end{frame}
\begin{frame}[label={sec:org0495dea}]{Order of accuracy : Results}
\begin{figure}[htbp]
\centering
\includegraphics[width=0.6\textwidth]{code/ooa.pdf}
\caption{Schemes exhibit different orders of accuracy}
\end{figure}
\end{frame}
\begin{frame}[label={sec:org623f32c}]{Order of accuracy : Results}
\begin{table}[htbp]
\caption{\label{tab_sym_snake_params}
Order of accuracy for different schemes}
\centering
\begin{tabular}{lr}
\toprule
Scheme & \(n(f(x))\)\\
\midrule
Euler & 1\\
Midpoint & 2\\
RK4 & 4\\
Verlet* & ?\\
\bottomrule
\end{tabular}
\end{table}
\end{frame}
\begin{frame}[label={sec:orgd702f37}]{Energy preserving scheme}
\begin{itemize}
\item Give definition of harmonic oscillator here too (just picture and equations)
\end{itemize}
\end{frame}
\begin{frame}[label={sec:org26d9d8b}]{Summary : give properties in a table}
\end{frame}
\end{document}
