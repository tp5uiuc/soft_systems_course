% Created 2019-02-04 Mon 17:25
% Intended LaTeX compiler: pdflatex
\documentclass[presentation]{beamer}
\usepackage[utf8]{inputenc}
\usepackage[T1]{fontenc}
\usepackage{graphicx}
\usepackage{grffile}
\usepackage{longtable}
\usepackage{wrapfig}
\usepackage{rotating}
\usepackage[normalem]{ulem}
\usepackage{amsmath}
\usepackage{textcomp}
\usepackage{amssymb}
\usepackage{capt-of}
\usepackage{hyperref}
\usepackage{awesomebox}
\usepackage{booktabs}
\usepackage{placeins}
\usepackage{siunitx}
\usepackage{minted}
\usetheme[progressbar=frametitle,block=fill]{metropolis}
\usetheme{default}
\author{\emph{Tejaswin Parthasarathy}, Mattia Gazzola}
\date{\today}
\title{Scientific computing in Python}
\subtitle{ME498: Comp. modeling \& optimization}
\hypersetup{
 pdfauthor={\emph{Tejaswin Parthasarathy}, Mattia Gazzola},
 pdftitle={Scientific computing in Python},
 pdfkeywords={},
 pdfsubject={},
 pdfcreator={Emacs 27.0.50 (Org mode 9.2)},
 pdflang={English}}
\begin{document}

\maketitle

\section{\texttt{numpy}}
\label{sec:orgcab7e74}
\begin{frame}[label={sec:org3d4c8d2},fragile]{\texttt{numpy} package \footnote{\href{https://www.numpy.org/}{numpy}}}
 \begin{itemize}
\item High-performance vector, matrix and higher-dimensional data structures for
\texttt{Python}
\item Shares a lot of similarity and differences (syntactically and semantically)
with \texttt{MATLAB} \footnote{\href{https://docs.scipy.org/doc/numpy/user/numpy-for-matlab-users.html}{numpy for Matlab users}}
\item Vectors, matrices and higher-dimensional data sets are \emph{(nd) arrays} in \texttt{numpy}
(there is also the \texttt{matrix} class, but it is being phased out)
\item Standard import---\texttt{import numpy as np}
\end{itemize}
\end{frame}
\begin{frame}[label={sec:orgd37e6fd},fragile]{Simple array creation in \texttt{numpy}}
 \note{:B\_note:
\begin{itemize}
\item Show the documentaion and how to browse it
\item Show as a demonstration\ldots{}
\end{itemize}}
\begin{itemize}
\item \texttt{v = np.array([1,2,3,4])} creates a vector (argument : list)
\item \texttt{M = np.array([[1, 2], [3, 4]])} creates a matrix (argument : nested list)
\item \texttt{type(v), type(M)} both return \texttt{numpy.ndarray}
\item The difference lies in the \alert{shape} seen using \texttt{v.shape/M.shape}\ldots{}
\item Alternatively use function \texttt{np.shape(v)}
\item Arrays can also have different data types, seen using \texttt{v.dtype}
\begin{itemize}
\item \texttt{npint32}, \texttt{np.float32}, \texttt{np.float64} (default)\ldots{}
\end{itemize}
\item \texttt{M = np.array([[1, 2], [3, 4]], dtype=int)}
\end{itemize}
\end{frame}
\begin{frame}[label={sec:org742b32d},fragile]{Array generating functions}
 \note{:B\_note:
\begin{itemize}
\item Show as a demonstration\ldots{}
\end{itemize}}
\begin{itemize}
\item \texttt{v = np.arange(0,11,2)} gives ranges similar to \texttt{Python's range}
\item \texttt{v = np.arange(0, 11, 0.1)} is also valid! (gives step of 0.1)
\item \texttt{v = np.linspace(0, 1, 3)} creates a linearly spaced vector [0, 0.5, 1.]
\item Also have \texttt{logspace, geomspace} for other progressions
\item Multidimensional array creation using \texttt{meshgrid, ndgrid} and others
\item Other useful ones are \texttt{ones} (which generates matrix with all 1), \texttt{zeros}
(simliar) and \texttt{eye} (identity matrix)
\end{itemize}
\end{frame}

\begin{frame}[label={sec:org7e36254},fragile]{Random numbers}
 \note{:B\_note:
\begin{itemize}
\item Show as a demonstration
\item np.random.rand(5,5)\ldots{}same syntax for randn too..
\item np.random.randint(1,5,size=(2,4))
\end{itemize}}
\begin{itemize}
\item We'll work extensively with random numbers, so let's see what \texttt{numpy} has
to offer
\item \texttt{np.random.rand(<shape>)} gives random floats from uniform dist. in [0, 1)
\item \texttt{np.random.randn(<shape>)} gives random floats from univariate normal
dist. with \(\mu = 0\) and \(\sigma = 1\)
\item \texttt{np.random.randint(low, high, size)} gives random ints from discrete uniform dist.
in \texttt{[low, high)}
\end{itemize}
\end{frame}
\begin{frame}[label={sec:org97334c7},fragile]{Indexing}
 \note{:B\_note:
\begin{itemize}
\item Show as a demonstration of all
\item A = np.random.randn(5,5); A[2:3,:] is slicing
\item Fancy indexing: idx\_rows = [1,5]; idx\_cols = [1,2,4]; A[idx\_rows, idx\_cols]
\item mask\_row = [True, False, False, False, True]; A[mask\_row]. Compare with
A[idx\_rows, :] above
\end{itemize}}
\begin{itemize}
\item Slicing works for \texttt{numpy} arrays too, across any dimension!
\item Fancy indexing
\begin{itemize}
\item Extends slicing to be more useful + practical
\end{itemize}
\item Masking
\begin{itemize}
\item Bools to \emph{mask} what is not necessary
\item Useful with conditional functions (e.g. \texttt{x < 5})
\end{itemize}
\item Reshaping using \texttt{np.reshape} changes index access
\end{itemize}
\end{frame}

\begin{frame}[label={sec:orgdfdf9b7},fragile]{Linear Algebra}
 \note{:B\_note:
\begin{itemize}
\item In the demo construct structure A with 1:25 using \texttt{A =
      np.linspace(1.0,25.0,25)} and then do \texttt{A = A.reshape(5,5)}
\item Use \texttt{np.arange} for constructing \texttt{v=np.arange(0, 5)}
\item Show that \texttt{A*v} does element wise onlt
\end{itemize}}

\begin{itemize}
\item Scalar operations on an array \texttt{A}
\begin{itemize}
\item \texttt{A + 2}, \texttt{A * 2} , \texttt{A ** 2} \ldots{}
\end{itemize}
\item Element wise operations on an array \texttt{A}
\begin{itemize}
\item \texttt{A * A}, \texttt{A / A} \ldots{}
\item What do you get when you do \texttt{A*v}? \alert{DEMO}
\end{itemize}
\item Matrix algebra on an array \texttt{A}
\begin{itemize}
\item \texttt{np.dot(A, v)} or \texttt{A.dot(v)} or simply \texttt{A@v}
\item Shape needs to be compliant! \texttt{numpy} also has broadcasts that is useful,
but is confusing and so is not covered here \ldots{}
\end{itemize}
\end{itemize}
\end{frame}

\begin{frame}[label={sec:org8e0f4f4},fragile]{Practice}
 \begin{block}{Please attempt}
\begin{itemize}
\item \texttt{12\_numpy\_library.ipynb}
\item For a more extensive tutorial: \url{https://github.com/donnemartin/data-science-ipython-notebooks\#numpy}
\end{itemize}
\end{block}
\end{frame}
\note{:B\_note:
\begin{itemize}
\item Access to \url{https://github.com/donnemartin/data-science-ipython-notebooks}
especially the numpy section
\end{itemize}}

\section{Genetic algorithm using \texttt{numpy}}
\label{sec:orgf2bee9b}
\begin{frame}[label={sec:org2aec3b2}]{Schematic}
\footnotesize
\begin{figure}[htbp]
\centering
\includegraphics[width=1.0\textwidth]{images/ga_schematic.png}
\caption{Schematic of an evolutionary search algorithm}
\end{figure}
\end{frame}
\end{document}
